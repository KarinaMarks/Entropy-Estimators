\documentclass{article}
\usepackage[utf8]{inputenc}
\title{Statistical Inference for Entropy}
\author{Karina Marks}

\usepackage{amsmath, amsfonts, graphicx, listings, booktabs, amstext}
\usepackage[format=plain,
            textfont=it]{caption}

\newtheorem{theorem}{Theorem}
\newtheorem{remark}{Condition}

\begin{document}
\section{Kozachenko-Leonenko Estimator}

\subsection{History}

This estimator was first introduced by L.Kozachenko and N.Leonenko, in 1987, where they first published the article \textit{Sample Estimate of the Entropy of a Random Vector}, in the paper \textit{Problems of Information Transmission}. Using the nearest neighbour method, they created a simple estimator for the Shannon entropy of an absolutely continuous random vector from a independent sample of observations, to then establish conditions under which we have asymptotic unbiasedness and consistency.

Since then, there has been major developments in the estimator; firstly in 2007, N.Leonenko, L.Pronzato, V.Savani, proposed a similar alternative to this estimator in their paper \textit{a Class of Renyi Information Estimators for Mulitdimensional densities}, this time using the k-nearest neighbour method, to consider estimators for the R\'enyi and Tsallis entropies. Then as the order of these entropies $q \to 1$, they defined the k-nearest neighbour estimator for the Shannon entropy, where k is fixed, and these estimators (under less rigorous conditions) are both consistent and asymptotically unbiased.

Also, the use of a fixed k has been backed up by a more recent paper in 2016, by S.Delattre and N.Fournier, \textit{On the Kozachenko-Leonenko Entropy Estimator}, which is a detailed study of the bias and variance of this estimator, using a fixed k. Subsequently finding that, in higher dimensions, the bias can be expressed in terms of $N^{-\frac{2}{d}}$; thus, leading to the development of explicit asymptotic confidence intervals.

Moreover, in 2016, a new idea was proposed by T.Berrett, R.Samsworth and M.Yuan, written in \textit{Efficient Mulitvariate Entropy Estimation via k-Nearest Neighbour Distances}; that the value chosen for $k$, depends upon the sample size $N$. Also, this idea is then extended to a new estimator; ''formed as a weighted average if Kozachenko-Leonenko estimators for different values of k''. I will not be exploring this new estimator in depth; however, the understanding of the value of $k$ depending on $N$ will be examined in detail.

\subsubsection{Estimator with k=1}

Firstly, I considered an article \textit{On Statistical Estimation of Entropy of Random Vector} (N.Leonenko and L.Kozachenko, 1987), which considers estimating the Shannon entropy of an absolutely continuous random sample of independent observations, with unknown probability density $f(x), x \in \mathbb{R}^{d}$. As $f(x)$ is unknown this is not easily estimated accurately for a random sample, and by just estimating the density $\hat{f}(x)$ to replace the actual density $f(x)$ in the formula for the entropy we get highly restrictive consistency conditions. 

Therefore, the following estimator was proposed for the Shannon entropy of a random sample $X_{1}, X_{2}, ..., X_{N}$ of d-dimensional observations;
\begin{equation}
H_{N} = d \log(\bar{\rho } ) + \log (c(d)) + \log (\gamma) + \log (N-1)
\end{equation}
where $c(d) = \frac{\pi^{\frac{d}{2}}}{\Gamma ( \frac{d}{2} + 1 )}$ is the volume of the d-dimensional unit ball, the Euler constant is $\log (\gamma) = \exp \left[ - \int_{0}^{\infty} e^{-t} \log(t) dt \right] = -\Psi(1)$ and $\bar{\rho} = \left[ \prod_{i=1}^{N} \rho_{i} \right]^{\frac{1}{N}}$, with $\rho_{i}$ the nearest neighbour distance from $X_{i}$ to another member of the sample $X_{j}$, $i \neq j$. 

It is important to note that one can write the Euler constant $-\Psi(1) = \log (\exp(-\Psi(1))) = \log (\frac{1}{\exp(\Psi(1))})$, this notation is what is used in the latter papers, so it is useful to introduce it here. $\Psi(x)$ is the Digamma function, and when $x=1$, this is just the Euler constant. Thus this estimator can be written if the form;
\begin{align}
H_{N} &= \log(\bar{\rho}^{d} ) + \log (c(d)) - \Psi(1)  + \log (N-1) \nonumber \\
&= \log \left( \left[ \prod_{i=1}^{N} \rho_{i} \right]^{\frac{d}{N}} \right) \log( c(d) (N-1)) + \log \left(\frac{1}{\exp(\Psi(1))}\right) \nonumber \\
&= \frac{1}{N} \sum_{i=1}^{N} \log( \rho_{i}^{d} ) + \log \left( \frac{c(d) (N-1)}{ \exp(\Psi(1))} \right) \nonumber \\
&= \frac{1}{N} \sum_{i=1}^{N}\log(\rho_{i}^{d}) + \frac{1}{N} \sum_{i=1}^{N} \log \left( \frac{c(d) (N-1)}{\exp(\Psi(1))}\right) \nonumber \\
&= \frac{1}{N} \sum_{i=1}^{N} \log \left( \frac{\rho_{i}^{d} c(d) (N-1)}{\exp(\Psi(1))}\right) \label{Est_k=1}
\end{align}

Under some strong conditions on the density function, this estimator is asymptotically unbiased and a consistent estimator for the Shannon entropy. 

The estimator here is in a simple form, which is later developed into something more sophisticated, using the nearest neighbour method, but considering larger values of $k$ (here $k=1$). This estimator is developed so that the consistency and asymptotic unbias of the estimator holds under less constrained conditions.


\subsubsection{Estimator with k fixed}

The next paper I am exploring on estimation is \textit{a Class of Renyi Information Estimators for Mulitdimensional densities} (N.Leonenko, L.Pronzato, V.Savani, 2007), which looks at estimating the R\'enyi ($H_{q}^{*}$) and Tsallis ($H_{q}$) entropies, when $q \neq 1$, and the Shannon ($\hat{H}_{N, k, 1}$) entropy. Where these are taken for a random vector $X \in \mathbb{R}^d$ with density function $f(x)$, by using the kth nearest neighbour method, with a fixed values of k. 

For the R\'enyi and Tsallis entropies, this is achieved by considering the integral  $I_{q} = \int_{\mathbb{R}^d} f^q (x) dx$, and generating its estimator, which is defined as $\hat{I}_{N, k, q} = \frac{1}{N} \sum_{i=1}^{N} (\zeta_{N, k, q})^{1-q}$. Where, $\zeta_{N, k ,q} = (N-1)C_{k}V_{d}(\rho_{k, N-1}^{(i)})^d$,  $V_{d} = \frac{\pi^{\frac{d}{2}}}{\Gamma(\frac{d}{2} + 1 )}$ is the volume of d-dimensional unit ball, $C_{k} = \left[ \frac{\Gamma(k)}{\Gamma(k+1-q)} \right]^{\frac{1}{1-q}}$ and $\rho_{k, N-1}^{(i)}$ is the kth nearest neighbour distance from the observation $X_{i}$ to some other $X_{j}$.

The estimator $\hat{I}_{N, k, q}$, provided $q>1$ and $I_{q}$ exists - and for any $q \in (1, k+1)$ if f is bounded - is thus found to be an asymptotically unbiased estimator for $I_{q}$. Also, provided  $q>1$ and $I_{2q-1}$ exists -  and for any $q \in (1, \frac{k+1}{2})$, when $k \geq 2$ if f is bounded - $\hat{I}_{N, k, q}$ is thus a consistent estimator for $I_{q}$. Moreover, by simple formulas both the R\'enyi and Tsallis entropies can be written in terms of this estimated value; 
\begin{align}
\hat{H}_{q}^{*} &= \frac{1}{1-q} log(\hat{I}_{N, k, q}) \\
\hat{H}_{q} &= \frac{1}{q-1} (1 - \hat{I}_{N, k, q})
\end{align}
thus, under the latter conditions, provide consistent estimates of these entropies as $N \to \infty$.

Furthermore, this paper goes on to discuss an estimator for the Shannon entropy, $H_{1}$ by taking the limit of the estimator for the Tsallis entropy, $\hat{H}_{N, k, q}$ as $q \to 1$, again with a fixed value of $k$. This estimator is similar to that discussed before by Leonenko in his 2004 paper, equation \ref{Est_k=1}; however, it is now extended from the nearest neighbour to the kth nearest neighbour;
\begin{equation}
\hat{H}_{N, k, 1} =  \frac{1}{N} \sum_{i=1}^{N} \log (\xi_{N, i, k})
\end{equation} 
where $\xi_{N, i, k} = (N-1)\exp[-\Psi(k)]V_{d}(\rho_{k, N-1}^{(i)})^{d}$, with $V_{d}$ and $\rho_{k, N-1}^{(i)}$ defined as in the estimation of $I_{q}$ and the digamma function $\Psi(z) = \frac{\Gamma'(z)}{\Gamma(z)}$. The digamma function at $k=1$ is given by $\Psi(1) = \log(\gamma)$, the Euler constant, which was used for the $k=1$ version of this estimator. Under the following less restrictive conditions; f is bounded, $I_{q_{1}}$ exists for some $q_{1} > 1$; then $H_{1}$ exists and the estimator $\hat{H}_{N, k, 1}$ is a consistent estimator for the Shannon entropy.

Extend this to also include paper 3


\subsubsection{Estimator with k=k(n)}

Paper 4

\subsection{Focus of this Paper}

I now wish to more explicitly introduce the Kozachenko-Leonenko estimator of the entropy H. Let $X_{1}, X_{2}, ... ,X_{N}$, $N \geq 1$ be independent and identically distributed random vectors in $\mathbb{R}^{d}$, and denote $\|.\|$ the Euclidean norm on $\mathbb{R}^{d}$.
 
\begin{itemize}

\item For $i = 1, 2, ..., N$, let $X_{(1), i}, X_{(2), i}, .., X_{(N-1), i}$ denote an order of the $X_{k}$ for $k = \{1, 2, ..., N\} \setminus \{i\}$, such that $\| X_{(1), i} - X_{i}\| \leq \cdots \leq \|  X_{(N-1), i} - X_{i}\| $. Let the metric $\rho$, defined as;
\begin{equation} \label{Rho}
\rho_{(k), i} = \| X_{(k), i} - X_{i}\|
\end{equation} denote the kth nearest neighbour or $X_{i}$.

\item  For dimension d, the volume of the unit d-dimensional Euclidean ball is defined as;
\begin{equation} \label{Volume}
V_{d} = \frac{\pi^\frac{d}{2}}{\Gamma(1 + \frac{d}{2})}
\end{equation}

\item For the kth nearest neighbour, the digamma function is defined as;
\begin{equation} \label{Psi}
\Psi(k) = -\gamma + \sum_{j=1}^{k-1} \frac{1}{j}
\end{equation}
where $\gamma = 0.577216$ is the Euler-Mascheroni constant (where the digamma function is chosen so that $\frac{e^{\Psi(k)}}{k}\to1$ as $k \to \infty$).

\end{itemize} Then the Kozachenko-Leonenko estimator for entropy, H, is given by;
\begin{equation} \label{KLest}
\hat{H}_{N, k} = \frac{1}{N} \sum_{i=1}^{N} log \left[ \frac{\rho_{(k),i}^{d} V_{d} (N-1)}{e^{\Psi(k)}} \right]
\end{equation} where, $\rho_{(k),i}^{d}$ is defined in (\ref{Rho}), $V_{d}$ is defined in (\ref{Volume}) and $\Psi(k)$ is defined in (\ref{Psi}).This estimator for entropy, when $d \leq 3$, under a wide range of k and some regularity conditions, satisfies some  theorems.



Theorem \ref{paper4_T5} holds, according to the central limit theorem, on the estimator for entropy $\hat{H}_{N, k}$;
\begin{equation}
\frac{\hat{H}_{N, k} - \mathbb{E}{\hat{H}_{N, k}}}{\sqrt{Var(\hat{H}_{N, k})}} \xrightarrow{d} N(0, \sigma^2) \nonumber
\end{equation}
By Theorem \ref{paper4_T4}, we can assume that $Var(\hat{H}_{N, k}) = \frac{Var(\log f(x))}{N} \approx \frac{1}{N}$, as for large $N$, the variance of the logarithm of the density function stays constant. Thus, the left side of the central limit theorem above can be written as;
\begin{align*}
\frac{\hat{H}_{N, k} - \mathbb{E}{\hat{H}_{N, k}}}{\sqrt{Var(\hat{H}_{N, k})}} &= \sqrt{N}(\hat{H}_{N, k} - \mathbb{E}{\hat{H}_{N, k}}) \\
&= \sqrt{N}[(\hat{H}_{N, k} - H) + (H - \mathbb{E}{\hat{H}_{N, k}})] \\
&= \sqrt{N}(\hat{H}_{N, k} - H) + \sqrt{N}(H - \mathbb{E}{\hat{H}_{N, k}})
\end{align*}
and as $N \to \infty$ this tends to the normal distribution, $N(0, \sigma^2)$. So we can say that $\sqrt{N}(\hat{H}_{N, k} - H) \xrightarrow{d} N(0, \sigma^2)$ while $\sqrt{N} (H - \mathbb{E}{\hat{H}_{N, k}}) \to \sigma^2$, which is equivalent to the properties stated in Theorem \ref{paper4_T5}.

Later, I will further discuss this estimator for the specific dimensions $d=1$ and $d=2$; however, it is important to note that for larger dimensions this estimator is not accurate. When $d=4$, equations (\ref{est_dist}) and (\ref{est_consist}) no longer hold but the estimator $\hat{H}_{N, k}$, defined by (\ref{KLest}), is still root-N consistent, provided k is bounded. Also, when $d \geq 5$ there is a non trivial bias, regardless of the choice of $k$. There is a new proposed estimator, formed as a weighted average of $\hat{H}_{N, k}$ for different values of $k$, where $k$ depends on the choice of $N$, explored in PAPER 4 (TODO reference). 

Moreover, this paper focuses only on distributions for $d \leq 3$, more specifically, I will first be considering samples from 1-dimensional distributions, $d=1$. Therefore, the volume of the 1-dimensional Euclidean ball is given by $V_{1} = \frac{\pi^{\frac{1}{2}}}{\Gamma (\frac{3}{2})} = \frac{\sqrt{\pi}}{\frac{\sqrt{\pi}}{2}} = 2$. Hence the Kozachenko-Leonenko estimator is of the form;
\begin{equation} \label{KLest_d=1}
\hat{H}_{N, k} = \frac{1}{N} \sum_{i=1}^{N} log \left[ \frac{2\rho_{(k),i}(N-1)}{e^{\Psi(k)}} \right]
\end{equation}
 Later, I will be considering samples from 2-dimensional distributions; thus, $d=2$ and the volume of the 2-dimensional Euclidean ball is given by $V_{2} = \frac{\pi^{\frac{2}{2}}}{\Gamma (2)} = \frac{\pi}{1} = \pi$. Hence, the estimator takes the form;
\begin{equation} \label{KLest_d=2}
\hat{H}_{N, k} = \frac{1}{N} \sum_{i=1}^{N} log \left[ \frac{\pi \rho_{(k),i}^{2} (N-1)}{e^{\Psi(k)}} \right]
\end{equation}


\end{document}