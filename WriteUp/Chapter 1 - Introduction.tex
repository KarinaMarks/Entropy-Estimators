\documentclass{article}
\usepackage[utf8]{inputenc}
\title{Chapter 2 - Introduction}
\author{Karina Marks}

\usepackage{amsmath, amsfonts, graphicx, listings, booktabs, amstext}
\usepackage[format=plain,
            textfont=it]{caption}

\newtheorem{theorem}{Theorem}
\newtheorem{remark}{Condition}

\begin{document}

\maketitle

\section{Abstract}


\section{Entropy}

Entropy $H(S)$, can be thought of as a representation of the average information content of an observation; sometimes referred to as a measure of unpredictability or disorder. 

''$H(S)$ is the quantity of surprise you should feel upon reading the result of a measurement'' (Fraser and Swinney, 1986) \cite{entdef1}. Thus the ''entropy of S can be seen as the uncertainty of S'' \cite{paper7}.

\subsection{Shannon Entropy}
The Shannon entropy of a random vector X with density function f is given by;
\begin{align} 
H &= - \mathbb{E} \{log(f(x))\} \nonumber \\
&= - \int_{x : f(x) > 0} f(x) log(f(x)) dx \nonumber \\
&= - \sum_{x \in \mathbb{R}^{d}} f(x) log(f(x)) \label{ShaEnt}
\end{align} 

\subsection{ R\'enyi and Tsallis Entropy}
These entropies are for the order $q \neq 1$ and the construction of them relies upon the generalisation of the Shannon entropy \ref{ShaEnt}. For a random vector $X \in \mathbb{R}^d$ with density function f, we define;

R\'enyi entropy
\begin{align} 
H_{q}^{*} &= \frac{1}{1-q} log \left( \int_{\mathbb{R}^d} f^q (x) dx \right) \quad  \quad (q \neq 1) \label{RenEnt} \\
&=  \frac{1}{1-q} log \left( \sum_{x \in \mathbb{R}^{d}} f^q (x) \right) \nonumber 
\end{align}

Tsallis entropy
\begin{align} 
H_{q} &= \frac{1}{q-1} \left(1 - \int_{\mathbb{R}^d} f^q (x) dx \right)  \quad  \quad (q \neq 1) \label{TsaEnt} \\
&=  \frac{1}{q-1} \left(1 - \sum_{x \in \mathbb{R}^d} f^q (x) \right) \nonumber 
\end{align}

When the order of the entropy $q \to 1$, both the R\'enyi, (\ref{RenEnt}), and Tsallis, (\ref{TsaEnt}), entropies tend to the Shannon entropy, (\ref{ShaEnt}), this is a special case for when $q=1$. 

\end{document}